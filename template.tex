\documentclass[legalpaper, oneside, final]{scrartcl}

\usepackage[legalpaper, top=1in, bottom=1in]{geometry}
\usepackage{titlesec} % Allows creating custom \section's
\usepackage{marvosym} % Allows the use of symbols
\usepackage{tabularx,colortbl,multirow} % Advanced table configurations
\usepackage[T1]{fontenc}
\usepackage[default]{gillius}
\usepackage{paracol}
\usepackage{enumitem}
\usepackage{microtype} % To enable letterspacing
\usepackage{fontawesome5}
\usepackage{keycommand}
\usepackage{layouts}
\usepackage{import}
\import{utilities/}{setup.tex}
\import{utilities/}{colors.tex}
\import{utilities/}{hyperref.tex}
\import{components/}{items.tex}
\import{components/}{datelist.tex}
\import{components/}{entrylist.tex}
\import{components/}{entry_work.tex}
\import{components/}{entry_project.tex}
\import{components/}{entry_education.tex}
\import{components/}{entry_volunteering.tex}
\import{components/}{properties-table.tex}

%----------------------------------------------------------------------------------------

\begin{document}

\begin{center} % Center everything in the document

%----------------------------------------------------------------------------------------
%	HEADER SECTION
%----------------------------------------------------------------------------------------

{\fontsize{36}{36} \selectfont\scshape{Muhammad As-Sawalhy}}

\vspace{0.4cm}

{\Large\Letter}
\href{mailto:MuhammadSawalhy@gmail.com} {MuhammadSawalhy@gmail.com}
{\large\textperiodcentered}
{\faPhone} \href{tel:+201096390741}{+201096390741}

\faGithub
\href{https://github.com/MuhammadSawalhy}
{@MuhammadSawalhy}
{\large\textperiodcentered}
\faLinkedin
\href{https://linkedin.com/in/MuhammadSawalhy}{@MuhammadSawalhy}

{\faMapMarker}
Abou-Kabir {\large\textperiodcentered} Ash-Sharkia {\large\textperiodcentered} Egypt

\vspace{0.4cm}

%----------------------------------------------------------------------------------------
%	WORK EXPERIENCE
%----------------------------------------------------------------------------------------

\section{Work Experience}

\begin{entrylist}

\work[
    period=August 2021 --- Present,
    employer=SciCave team,
    title=Frontend developer,
    skills=ReactJS \& Django
]{
    \begin{items}
    \item Led the front-end development.
    \item Designed prototypes using Figma.
    \item Deployed web apps using docker and ssh to access the remote server.
    \item Helped with some tasks at the back-end side with Django. % TODO: find a better way to do this
    \end{items}
}

\work[
    period=2016 summer,
    title=Weighbridge Operator,
]{
    \begin{items}
    \item Operated the weighbridge.
    \item Issued tickets.
    \end{items}
}

\end{entrylist}

%-------------------------------------------------------------------------------------
%	Honors & Awards
%-------------------------------------------------------------------------------------

\section{Honors \& Awards}

\begin{datelist}
    \dateentry {2022}
    {Recognized as best member of \linkage{https://www.linkedin.com/company/zagengfamily/}{ZagEng}'s IT Committee.}

    \dateentry {2019}
    {Honored by \linkage{http://www.mtc.edu.eg/mtcwebsite/Index.aspx}{MTC (Military Technical College - Egypt)} for being one of the top high school students.}
\end{datelist}

%----------------------------------------------------------------------------------------
% Activities & Volunteering
%----------------------------------------------------------------------------------------

\section{Activities \& Volunteering}

\begin{entrylist}

    \volunteering[
        period=November 2021 --- present,
        role=IT Committee Member,
        organization=\linkage{https://www.linkedin.com/company/zagengfamily/}{ZagEng Team},
    ]{
        \begin{items}
        \item Recognized as the best member of the committee.
        \item Maintained a telegram bot for the 1st year civil engineering students.
        \item Automated some of the processes and routines with python code.
        \item Built a web app with Django to automate sending messages to multiple bots at once.
        \item Automated sending messages at specific times daily with Integromat.
        \end{items}
    }

\end{entrylist}

%-------------------------------------------------------------------------------------
%	EDUCATION
%-------------------------------------------------------------------------------------

\section{Education}

\begin{entrylist}

    \education[
        period={2019 --- 2024, expected},
        degree=Bachelor of Engineering,
        major=Electrical Engineering Department,
        authority=Zagazig University,
    ]{
        \begin{items}
        \item Got excellent (A-) for the past 5 semesters.
        \end{items}
    }

\end{entrylist}

%--------------------------------------------------------------------------------------
%	SKILLS
%--------------------------------------------------------------------------------------

\section{Skills}

\begin{tabularx}{0.95\textwidth}{ @{} >{\bfseries}p{2.4cm} | @{\hspace{6ex}} X}
    \parbox[t]{2.4cm}{Programming Languages}
    & JavaScipt, Python, Java, CPP, Shell Script, \LaTeX \\
    \\[-2mm] \hline \\[-2mm]

    \parbox{2.4cm}{Frontend}
    & Prototyping, Webpack, Monorepos, Testing \& Jest,
    ReactJS, Nodejs, JS, CSS, Sass, APIs \& RESTful, JSON \& Yaml \\
    \\[-2mm] \hline \\[-2mm]

    \parbox{2.4cm}{Tools}
    & Docker, Git, GNU tools, Vim \& Neovim, Tmux, Terminals, SSH \& GPG
\end{tabularx}

%-------------------------------------------
% Projects
%-------------------------------------------

\clearpage

\section{Projects}

\textcolor{note}{\emph{These are projects I am proud of. A picked samples not all...}}

\vspace{3mm}

% FIXME: paracol doesn't break page if it is too long
\begin{paracol}{2}

    \begin{entrylist}
        \project[
            name=Da7ee7 Bot Dashboard,
            period=2022,
            techs={Django, Javascript, CSS Bootstrap, HTML5},
            url=https://github.com/MuhammadSawalhy/da7ee7-bot-dashboard
        ]{
            Dashboard to control multiple telegram bots and send messages to all of them at once.
        }

        \project[
            name=Trafic light with ICs,
            period=2022,
        ]{
            This project was required by the faculty. A finite state machine (Moore machine) as a practice of the logic design course. I used basic logic gates such as \textsc{AND}, \textsc{OR}, \textsc{XOR}, and the story's hero \textsc{D flip-flop} which will maintain the state.
        }

        \project[
            name=BCD counter,
            period=2022,
        ]{
            This project was required by the faculty. A counter that counts from 0 to 9 and starts all over again from 0 was required to be designed and implemented using logic gates and the principles and concepts learnt from the logic design course.
        }

        \project[
            name=Mokhalasa web app,
            period=2021-2022,
            techs={ReactJs, Sass, Django, PostgresDB}
        ]{
            Web app for \linkage{https://mokalasah.com/}{Mokhalasa} collection agency to help manage their partners and clients and assign tasks to empolyees and keep track of how much debt they collected.
        }

        \project[
            name=Alif programming Language,
            url=https://aliflang.org/,
            period={2021 - Present {\scriptsize| Discontinuous,}},
            techs={Python, CSS, JavaScript, Shell Scripts}
        ]{
            Contributed to Alif programming language implement the testing scripts and add some tests to help detect bugs and problem at early stage. I also build a website for documentation and an online editor.
        }
        
    \end{entrylist}

    \switchcolumn

    \begin{entrylist}

        \project[
            name=Audio amplifier using a BJT (transistor),
            period=2021,
        ]{
            This project was part of a faulty course, which required us to amplify and increase the audio sound level so we can here low phone sound from an audio jack to a speaker which we can here meters away from the phone and speaker.
        }

        \project[
            name=ArabiJS Programming Language,
            period=2021,
            techs={JavaScript},
            url=https://github.com/arabi-js/arabi,
        ]{
            ArabiJS is a JavaScript arabization which uses Arabic keyword and translated builtin and standard APIs in addition to ability to translate third-party libraries which simple few lines of code.
        }

        \project[
            name=\TeX Math Parser \& Math Parser,
            period=2020 - 2021,
            url=https://www.npmjs.com/package/@scicave/math-latex-parser,
            techs={JavaScript, PEG.js}
        ]{
            A parser generated by PEG.js by providing the parsing rules in the PEG.js language. One parser for \LaTeX math expressions and another one for regular human readable math expressions.
        }

        \project[
            name=Plotto,
            period=2019-2021 {\scriptsize| Discontinuous},
            techs={Sass, JavaScript, Gulp, PugJS}
        ]{
            Graphing calculator to plot explicit mathematical functions and many other feature include adding point, drawing geometric shapes, and animation.

            \begin{items}
            \item Microsoft Store app: \url{https://bit.ly/plotto-win-app}
            \item Web app: \url{https://bit.ly/plotto-web-app}
            \end{items}
        }

        \project[
            name=GraphMaker website,
            period=2019,
            techs={HTML, CSS, JavaScript},
            url=https://graphmaker.netlify.app/,
        ]{
            A website of an app to view screenshots and download it.
        }
    \end{entrylist}

\end{paracol}

%----------------------------------------------------------------------------------------

\end{center}

\end{document}
